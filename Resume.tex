\documentclass[10pt, letterpaper]{article}

% Packages:
\usepackage[
    ignoreheadfoot, % set margins without considering header and footer
    top=2 cm, % seperation between body and page edge from the top
    bottom=2 cm, % seperation between body and page edge from the bottom
    left=2 cm, % seperation between body and page edge from the left
    right=2 cm, % seperation between body and page edge from the right
    footskip=1.0 cm, % seperation between body and footer
    % showframe % for debugging 
]{geometry} % for adjusting page geometry
\usepackage{titlesec} % for customizing section titles
\usepackage{tabularx} % for making tables with fixed width columns
\usepackage{array} % tabularx requires this
\usepackage[dvipsnames]{xcolor} % for coloring text
\definecolor{primaryColor}{RGB}{0, 79, 144} % define primary color
\usepackage{enumitem} % for customizing lists
\usepackage{fontawesome5} % for using icons
\usepackage{amsmath} % for math
\usepackage[
    pdftitle={Saksham Mishra's CV},
    pdfauthor={Saksham Mishra},
    pdfcreator={LaTeX with RenderCV},
    colorlinks=true,
    urlcolor=primaryColor
]{hyperref} % for links, metadata and bookmarks
\usepackage[pscoord]{eso-pic} % for floating text on the page
\usepackage{calc} % for calculating lengths
\usepackage{bookmark} % for bookmarks
\usepackage{lastpage} % for getting the total number of pages
\usepackage{changepage} % for one column entries (adjustwidth environment)
\usepackage{paracol} % for two and three column entries
\usepackage{ifthen} % for conditional statements
\usepackage{needspace} % for avoiding page brake right after the section title
\usepackage{iftex} % check if engine is pdflatex, xetex or luatex

% Ensure that generate pdf is machine readable/ATS parsable:
\ifPDFTeX
    \input{glyphtounicode}
    \pdfgentounicode=1
    % \usepackage[T1]{fontenc} % this breaks sb2nov
    \usepackage[utf8]{inputenc}
    \usepackage{lmodern}
\fi



% Some settings:
\AtBeginEnvironment{adjustwidth}{\partopsep0pt} % remove space before adjustwidth environment
\pagestyle{empty} % no header or footer
\setcounter{secnumdepth}{0} % no section numbering
\setlength{\parindent}{0pt} % no indentation
\setlength{\topskip}{0pt} % no top skip
\setlength{\columnsep}{0cm} % set column seperation
\makeatletter
\let\ps@customFooterStyle\ps@plain % Copy the plain style to customFooterStyle
\patchcmd{\ps@customFooterStyle}{\thepage}{
    \color{gray}\textit{\small Saksham Mishra - Page \thepage{} of \pageref*{LastPage}}
}{}{} % replace number by desired string
\makeatother
\pagestyle{customFooterStyle}

\titleformat{\section}{\needspace{4\baselineskip}\bfseries\large}{}{0pt}{}[\vspace{1pt}\titlerule]

\titlespacing{\section}{
    % left space:
    -1pt
}{
    % top space:
    0.3 cm
}{
    % bottom space:
    0.2 cm
} % section title spacing

\renewcommand\labelitemi{$\circ$} % custom bullet points
\newenvironment{highlights}{
    \begin{itemize}[
        topsep=0.10 cm,
        parsep=0.10 cm,
        partopsep=0pt,
        itemsep=0pt,
        leftmargin=0.4 cm + 10pt
    ]
}{
    \end{itemize}
} % new environment for highlights

\newenvironment{highlightsforbulletentries}{
    \begin{itemize}[
        topsep=0.10 cm,
        parsep=0.10 cm,
        partopsep=0pt,
        itemsep=0pt,
        leftmargin=10pt
    ]
}{
    \end{itemize}
} % new environment for highlights for bullet entries


\newenvironment{onecolentry}{
    \begin{adjustwidth}{
        0.2 cm + 0.00001 cm
    }{
        0.2 cm + 0.00001 cm
    }
}{
    \end{adjustwidth}
} % new environment for one column entries

\newenvironment{twocolentry}[2][]{
    \onecolentry
    \def\secondColumn{#2}
    \setcolumnwidth{\fill, 4.5 cm}
    \begin{paracol}{2}
}{
    \switchcolumn \raggedleft \secondColumn
    \end{paracol}
    \endonecolentry
} % new environment for two column entries

\newenvironment{header}{
    \setlength{\topsep}{0pt}\par\kern\topsep\centering\linespread{1.5}
}{
    \par\kern\topsep
} % new environment for the header

\newcommand{\placelastupdatedtext}{% \placetextbox{<horizontal pos>}{<vertical pos>}{<stuff>}
  \AddToShipoutPictureFG*{% Add <stuff> to current page foreground
    \put(
        \LenToUnit{\paperwidth-2 cm-0.2 cm+0.05cm},
        \LenToUnit{\paperheight-1.0 cm}
    ){\vtop{{\null}\makebox[0pt][c]{
        \small\color{gray}\textit{Last updated in June 2025}\hspace{\widthof{Last updated in June 2025}}
    }}}%
  }%
}%

% save the original href command in a new command:
\let\hrefWithoutArrow\href

% new command for external links:
\renewcommand{\href}[2]{\hrefWithoutArrow{#1}{\ifthenelse{\equal{#2}{}}{ }{#2 }\raisebox{.15ex}{\footnotesize \faExternalLink*}}}


\begin{document}
    \newcommand{\AND}{\unskip
        \cleaders\copy\ANDbox\hskip\wd\ANDbox
        \ignorespaces
    }
    \newsavebox\ANDbox
    \sbox\ANDbox{}

    \placelastupdatedtext
    \begin{header}
        \textbf{\fontsize{24 pt}{24 pt}\selectfont Saksham Mishra} \\[0.1cm]
        \normalsize\textit{Address: Gurugram, India} \\
        \vspace{0.3 cm}

        \normalsize
        \mbox{\hrefWithoutArrow{mailto:saksham2684@gmail.com}{\color{black}{\footnotesize\faEnvelope[regular]}\hspace*{0.13cm}saksham2684@gmail.com}}%
        \kern 0.25 cm%
        \AND%
        \kern 0.25 cm%
        \mbox{\hrefWithoutArrow{tel:+91-828-775-87-67}{\color{black}{\footnotesize\faPhone*}\hspace*{0.13cm}0828 775 87 67}}%
        \kern 0.25 cm%
        \AND%
        \kern 0.25 cm%
        \mbox{\hrefWithoutArrow{https://www.linkedin.com/in/saksham-m-1a2337290}{\color{black}{\footnotesize\faLinkedinIn}\hspace*{0.13cm}LinkedIn}}%
        \kern 0.25 cm%
        \AND%
        \kern 0.25 cm%
        \mbox{\hrefWithoutArrow{https://github.com/SakshamMishra2023}{\color{black}{\footnotesize\faGithub}\hspace*{0.13cm}GitHub}}%
        \kern 0.25 cm%
        \AND%
        \kern 0.25 cm%
        \mbox{\hrefWithoutArrow{https://sakshammishra2023.github.io/Portfolio_website/}{\color{black}{\footnotesize\faGlobe}\hspace*{0.13cm}Portfolio}}%
    \end{header}

    \vspace{0.3 cm - 0.3 cm}

    \section{Education}

        \begin{twocolentry}{
            
            
        \textit{July 2023 – Present}}
            \textbf{Indraprastha Institute of Information Technology, Delhi}

            \textit{B-Tech in Electronics and Communication Engineering}
        \end{twocolentry}

        \vspace{0.10 cm}
        \begin{onecolentry}
            \begin{highlights}
                \item \textbf{Coursework:} Computer Organization, Linear Algebra, Advanced Programming, Embedded Logic Design, Data Structures and Algorithms
            \end{highlights}
        \end{onecolentry}

        \vspace{0.2 cm}

        \begin{twocolentry}{


            
        \textit{2020 – 2022}}
            \textbf{Dhruva Public School, Delhi}

            \textit{Class-XII, PCM (92.2\% CBSE)}
        \end{twocolentry}

        \vspace{0.2 cm}

        \begin{twocolentry}{


            
        \textit{2013 – 2020}}
            \textbf{Greenwood Public School, Gurugram}

            \textit{Class-X, (90.2\% CBSE)}
        \end{twocolentry}

        \vspace{0.10 cm}
        

    \section{Technologies}

        \begin{onecolentry}
            \textbf{Expertise Area:} Web Development, IoT \& Micro-controllers, Object Oriented Programming
        \end{onecolentry}

        \vspace{0.2 cm}

        \begin{onecolentry}
            \textbf{Soft Skills:} Communication Skills, Critical Thinking, Self-Learning, Team Work
        \end{onecolentry}

        \vspace{0.2 cm}
        \begin{onecolentry}
            \textbf{Languages:} C/C++, Python, Java, Javascript, Verilog
        \end{onecolentry}

        \vspace{0.2 cm}

        \begin{onecolentry}
            \textbf{Technologies:} Git, SQL, VSCode, IntelliJ, Arduino IDE, KiCad, React, Figma 
        \end{onecolentry}


    \section{Volunteering}

        \begin{twocolentry}{
            
            \textit{Mar,25 -Jul'25}}
            \textbf{ Earth Saviours Foundation - Volunteering}
        \end{twocolentry}
        
        \vspace{0.10 cm}
        \begin{onecolentry}
            \begin{highlights}
                \item Collaborated with the NGO’s administration team to support and coordinate environment
related initiatives.
                \item Provided counseling and assistance to senior citizens, underprivileged individuals, and per
sons with disabilities residing in the Gurukul
                \item Volunteered at the onsite medical clinic, offering logistical support to patients and staff.
            \end{highlights}
        \end{onecolentry}


    \section{Projects}

        \begin{twocolentry}{
            
            
        \textit{Feb,24 – Apr,24}}
            \textbf{RISC – V Custom Assembler \& Simulator | \href{https://github.com/SakshamMishra2023/RISC_V_Assembler-Simulator.git}{github repo}}
        \end{twocolentry}

        \vspace{0.10 cm}
        \begin{onecolentry}
            \begin{highlights}
                \item A custom RISC V Assembler and Simulator written in C++ to convert RISC-V assembly code
into machine code (binary instructions) and vice-versa for execution on a RISC-V processor.
                \item Tech Stack: C++, RISC-V instruction set 
                \item Key-Impact: Strengthened understanding of system architecture and instruction pipelines.
                \item Team Size: 4
                \end{highlights} \end{onecolentry}

        \vspace{0.2 cm}

        \begin{twocolentry}{
            
            
        \textit{Oct,24 – Nov,24}}
            \textbf{Angry Bird Style Video-Game | \href{https://github.com/Sanchit-100/AngryGhosts.git}{github repo}}}
            
        \end{twocolentry}
        \textit{(Guide: Prof. Sambuddho Chakravarty)}

        \vspace{0.10 cm}
        \begin{onecolentry}
            \begin{highlights}
                \item Developed an angry-bird style game using LibGDX library of Java with 3 playable levels.
Described the in-game mechanisms using structured and behavioral UML diagrams.
                \item Tech Stack: LibGDX(Java), gradle, Box2D
                \item Key-Impact: Reinforced object-oriented game design principles and real-time physics simulation.
                \item Team Size: 2
            \end{highlights}
        \end{onecolentry}

        \vspace{0.2 cm}

        \begin{twocolentry}{
            
            
        \textit{Feb,24 –Apr,24}}
            \textbf{SpotSync – Parking Optimization App | \href{https://github.com/TheSlothThatCodes0/SpotSync.git}{github repo}}
        \end{twocolentry}

        \textit{(Guide: Prof. Rajiv Ratan Shah)}

        \vspace{0.10 cm}
        \begin{onecolentry}
            \begin{highlights}
                \item An application that optimizes parking space allocation in real time using live camera feeds .
Self Check-in and Check-out: Calculates parking fee by maintaining record of entry and exit
time through License plate recognition.
                \item Tech Stack: Python, React Native, OpenOCR, YOLO Model
                \item Key-Impact: Reinforced object-oriented game design principles and real-time physics simulation.
                \item Team Size: 5
            \end{highlights}
        \end{onecolentry}

        \vspace{0.2 cm}

        \begin{twocolentry}{
            
            
        \textit{Mar,24 – Apr,24}}
            \textbf{IOT Based RFID card Attendance System| \href{https://github.com/SakshamMishra2023/RFID-based-attendance-system.git}{github repo}}
        \end{twocolentry}

        \vspace{0.10 cm}
        \begin{onecolentry}
            \begin{highlights}
                \item Developed a smart attendance system leveraging RFID technology and the NodeMCU
ESP8266 microcontroller. This innovative project streamlines the process of attendance
tracking and enhances accuracy and efficiency.
                \item Tech Stack: ESP8266 Node-MCU, Arduino IDE, RFID
                \item Key-Impact: Eliminated manual tracking and ensured real-time, tamper-proof attendance records.
            \end{highlights}
        \end{onecolentry}

        \vspace{0.2 cm}

        \begin{twocolentry}{
            
            
        \textit{Jul,25 -Jul,25}}
            \textbf{Retail-Rush – Smart Gamified Retail Web App | \href{https://github.com/SakshamMishra2023/Retail-Rush.git}{github repo}}
        \end{twocolentry}

        \vspace{0.10 cm}
        \begin{onecolentry}
            \begin{highlights}
                \item Developed a responsive gamified retail web app using React.js, Vite, and Tailwind CSS to enhance user engagement and reduce product waste.Implemented efficient state management with React Hooks and modular design using mock data.Integrated dynamic UI elements like Vanta.js animations and SVG icons for an immersive shopping experience.
                \item Tech Stack : Tailwind CSS, React.Js
                \item Key-Impact : Increased user engagement potential with gamified incentives and real-time UI responsiveness.
                \item Team Size: 2
            \end{highlights}
        \end{onecolentry}

    
        

    \section{Positions of Responsibility}

        \begin{twocolentry}{
            
            
        \textit{Sep,2023 – Present}}

        \begin{highlights}
                \item \textf{Content Team: E cell (Entrpreneurial Club) } \end{highlights}
                \end{twocolentry}


         \begin{twocolentry}{

        \textit{Sep,24 – Dec,24}}
            \begin{highlights}
                \item \textf{ECE Newsletter Editor (Issue 09) } \end{highlights}
        \end{twocolentry}

        \begin{twocolentry}{
        \textit{Aug,24 – May,25}}
            \begin{highlights}
                \item \textf{Volunteer at Centre for Intelligent Product Development} \end{highlights}
        \end{twocolentry}

        \begin{twocolentry}{
        \textit{Jan,24 – Mar,24}}
            \begin{highlights}
                \item \textf{Content Team: E-summit 2023 } \end{highlights}
        \end{twocolentry}

        

    \section{Achievements and Awards}

        \begin{onecolentry}
            % Add your honors and awards here
            % Example format:
            % \textbf{Award Name} - Description or details about the award
            \begin{highlights}
                \item \textf{Finalist in Anveshan 3.0 (Intra-College Hackathon) | Position : Team Leader , Project : Spotsync}
                 \item \textf{Amazon Hack-on Season 5 : Among top 1400 teams out of 17,000+ teams}
            \end{highlights}

        \end{onecolentry}

    \section{Hobbies and Interests}

        \begin{onecolentry}
            % Add your hobbies and interests here
            % Example format:
            \begin{highlights}
                \item \textf{Writing Stories, Reading Books, Table Tennis}
                 \item \textf{Video Games, Dungeons \& Dragons}
            \end{highlights}

        \end{onecolentry}
    
    \vspace{0.40 cm}
    Declaration: The above information is correct to the best of my knowledge.

\end{document}
